\documentclass[12pt,letterpaper]{article}
\begin{document}

\title{COLLEGE OF COMPUTING AND INFORMATION SCIENCES\\ DEPARTMENT OF COMPUTER SCIENCE\\ SCHOOL OF COMPUTING AND INFORMATICS TECHNOLOGY\\}
\maketitle
\title{NAME:	      OKIRING PAUL                 MWESIGYE MICHAE          OKELLO JOHN PAUL\\
STUDENTS NUMBER:      214016765                    214019414                  215005517
REGISTRATION NUMBER:  14/U/13973/EVE               14/U/10401/EVE              15/U/11991/EVE
\maketitle

\section{INTRODUCTION}

     Model checking has emerged as a powerful method for the formal verification of programs. Temporal logics such as CTL (computational tree logic) and CTL* are widely used to specify programs because they are expressive and easy to understand. Given an abstract model of a program, a model checker (which typically implements the acceptance problem for a class of automata) verifies whether the model meets a given specification. A conceptually attractive method for solving the model checking problem is by reducing it to the solution of (a suitable subclass of) parity games. These are a type of two player infinite game played on a finite graph.
\section{PROBLEM STATEMENT}
Given a model of a system, exhaustively and automatically check whether this model meets a given specification. Typically, one has hardware or software systems in mind, whereas the specification contains safety requirements such as the absence of deadlocks and similar critical states that can cause the system to crash. Model checking is a technique for automatically verifying correctness properties of finite-state systems.
In order to solve such a problem algorithmically, both the model of the system and the specification are formulated in some precise mathematical language. To this end, the problem is formulated as a task in logic, namely to check whether a given structure satisfies a given logical formula. This general concept applies to many kinds of logics and suitable structures. A simple model checking problem is verifying whether a given formula in the propositional logic is satisfied by a given structure.

\section{BACKGROUND}
The satisfiability problem for branching-time temporal logics like CTL*, CTL and CTL+ has important applications in program specification and verification. Their computational complexities are known: CTL* and CTL+ are complete for doubly exponential time, CTL is complete for single exponential time. Some decision procedures for these logics are known; they use tree automata, tableaux or axiom systems.
Automata-theoretic approaches. As much as the introduction of CTL* has led to an easy unification of CTL and LTL, it has also proved to be quite a difficulty in obtaining decision procedures for this logic. The first procedure by Emerson and Sistla was automata-theoretic [ES84] and roughly works as follows. A formula is translated into a doubly-exponentially large tree automaton whose states are Hintikka-like sets of sets of sub formulas of the input formula.
This tree automaton recognizes a superset of the set of tree models of the input formula. It
is lacking a mechanism that ensures that certain temporal operators are really interpreted
as least fix points of certain monotone functions rather than arbitrary fix points.
Other approaches. Apart from these automata-theoretic approaches, a few deferent ones
have been presented as well. For instance, there is Reynolds' proof system for validity
[Rey01]. Its completeness proof is rather intricate and relies on the presence of a rule which
violates the sub formula property. In essence, this rule quantity over an arbitrary set of
atomic propositions. Thus, while it is possible to check a given tree for whether or not it is
a proof for a given formula, it is not clear how this system could be used in order to
find proofs for given formulas.


\section{MAIN OBJECTIVES}
 To show the connexions between the temporal logics CTL and / or CTL*, automata, and games.
\subsection{OBJECTIVES}
  \begin{enumerate}
    \item Representing CTL / CTL* as classes of alternating tree automata\\
    \item Inter-translation between CTL / CTL* and classes of alternating tree automata\\
    \item Using B¨uchi games and other subclasses of parity games to analyze the CTL / CTL* model checking problem\\
    \item Efficient implementation of model checking algorithms\\
    \item Application of the model checker to higher-order model checking\\
  \end{enumerate}
\section{Scope}
\section{Descriptions}
\section{Significances}
Advantages of the game-based approach. The game-theoretic framework uniformly treats the standard branching-time logics from the relatively simple CTL to the relatively complex CTL.
 It yields complexity-theoretic optimal results, i.e. satisfiability checking using this framework
is possible in exponential time for CTL and doubly exponential time for CTL+.
 Like the automata-theoretic approaches, it separates the characterization of satisfiability
through a syntactic object (a parity game) from the test for satisfiability (the problem
of solving the game). Thus, advances in the area of parity game solving carry over to
satisfiability checking. Like the tableaux-based approach, it keeps a very close relationship between the input formula and the structure of the parity game thus enabling feedback from a (counter-)model or applications in specification and verification. Satisfiability checking procedures based on this framework are implemented in the
MLSolver platform [FL10] which uses the high-performance parity game solver PG-Solver [FL09] as its algorithmic backbone.

\section{Methodologies}
The decision procedures for these logics are known;
\paragraph{ The tree automatas}
	a tree automaton is a type of state machine that deals with tree structures,rather than the strings of more conventional state machines.
\paragraph{ Tableaux}
	this is a decision method for propositional temporal logic that can be used to formalize reasoning about time and temporal relations.
\paragraph{Axiom systems}
	An axiomatic system is any set of axioms from which some or all axioms can be used in conjuction to logically derive theorems.
Due to the ability of temporal logic to allow us to make deductive arguments about not only what is,but what was , what will be and what will always be, this has led to the application of the logics to specifications and verification of reactive and concurrent programs and systems i.e games.
The key temporal patterns of importance in specifying such programs are:
\subparagraph{“Liviness” properties or eventualities }– which ensure a specific precondition is intially satisfied. Then a desiarable state is eventually reached.
\subparagraph{“Safety” or “Invariance” properties} – this ensures that a specific precondition is intially satisfied, then undesirable state will never occur.
\subparagraph{“Fairness” properties} -Fairness requires that in a system where several processes sharing resources are run concurrently , they must be treated fairly by the program.
\paragraph{Artificial Intelligence}
	Temporal reasoning has also been naturally combined with other well developed framework for AI such as the situation calculas(Pinto and Rieter 1995) and action theory ( Lamport 1994)

Traditionally methods such as temporal arguments in which the temporaldimensions is captured by augmenting each time variable proposition or predicate with an extra argument place to be filled by an expression designating a time.
\section{References}
\begin{list}{•}{•}
\item 1.Feidmann, O. (2013). Satisfiability Games For Branching-Time Logics. University of .\\
\item 2.Grumberg, O. (2008). 25 Years Of MOdel Checking. Grumberg, Orna, Veith, Helmut (Eds.).\\
\item 3.Monkowski. (2015, May 20). Temporal Logic. Retrieved April 18, 2017, from plato.stanford.edu \\
\item 4.Orna Kupferman, Moshe Y. Vardi, Pierre Wolper: An automata-theoretic approach to branchingtime model checking. J. ACM 47(2): 312-360 (2000).
http://dx.doi.org/10.1145/333979.333987 \\
\item 5.Pinto,D. Rieter,T. (1995) Knowledge Representation, Reasoning, and the Design of Intelligent Agents. Cambridge University Press\\


\end{list}
\end{document} 
